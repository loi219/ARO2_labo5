\documentclass[a4paper]{article} %format de la feuille + type de document https://en.wikibooks.org/wiki/LaTeX/Document_Structure#Document_classes
%packages nécessaire pour nos besoins
\usepackage[utf8]{inputenc}
\usepackage[T1]{fontenc}
\usepackage[english,french]{babel}
\usepackage{amsmath}
\usepackage{amssymb,amsfonts,textcomp}
\usepackage[usenames, dvipsnames]{color}
\usepackage[dvipsnames]{xcolor}
\usepackage{array}
\usepackage{hhline}
\usepackage{hyperref}
\usepackage[pdftex]{graphicx}
\usepackage{sectsty}
\usepackage{tcolorbox}
\usepackage{textcomp}
\usepackage{courier}
\usepackage[font={small,it}]{caption}
\usepackage{float}
\usepackage{graphicx}
\usepackage{caption}
\usepackage{tabularx}
\usepackage{multirow}% http://ctan.org/pkg/multirow
\usepackage{tikz}
\usepackage[top=25mm,bottom=25mm,right=25mm,left=25mm]{geometry} 
\usepackage[export]{adjustbox}
\usepackage{listings}
\usepackage{enumitem}


%Définition des couleurs
\definecolor{havelockBlue}{rgb}{0.004, 0.42, 0.73}
\definecolor{Monokaimagenta}{rgb}{0.86,0.08,0.24}
\definecolor{codegray}{rgb}{0.5,0.5,0.5}
\definecolor{backcolour}{rgb}{0.95,0.95,0.92}

\lstdefinelanguage[mips]{Assembler}{%
  % so listings can detect directives and register names
  alsoletter={.\$},
  % strings, characters, and comments
  morestring=[b]",
  morestring=[b]',
  morecomment=[l]\#,
  % instructions
  morekeywords={[1]abs,abs.d,abs.s,add,add.d,add.s,addi,addiu,addu,%
    and,andi,b,bc1f,bc1t,beq,beqz,bge,bgeu,bgez,bgezal,bgt,bgtu,%
    bgtz,ble,bleu,blez,blt,bltu,bltz,bltzal,bne,bnez,break,c.eq.d,%
    c.eq.s,c.le.d,c.le.s,c.lt.d,c.lt.s,ceil.w.d,ceil.w.s,clo,clz,%
    cvt.d.s,cvt.d.w,cvt.s.d,cvt.s.w,cvt.w.d,cvt.w.s,div,div.d,div.s,%
    divu,eret,floor.w.d,floor.w.s,j,jal,jalr,jr,l.d,l.s,la,lb,lbu,%
    ld,ldc1,lh,lhu,li,ll,lui,lw,lwc1,lwl,lwr,madd,maddu,mfc0,mfc1,%
    mfc1.d,mfhi,mflo,mov.d,mov.s,move,movf,movf.d,movf.s,movn,movn.d,%
    movn.s,movt,movt.d,movt.s,movz,movz.d,movz.s,msub,msubu,mtc0,mtc1,%
    mtc1.d,mthi,mtlo,mul,mul.d,mul.s,mulo,mulou,mult,multu,mulu,neg,%
    neg.d,neg.s,negu,nop,nor,not,or,ori,rem,remu,rol,ror,round.w.d,%
    round.w.s,s.d,s.s,sb,sc,sd,sdc1,seq,sge,sgeu,sgt,sgtu,sh,sle,%
    sleu,sll,sllv,slt,slti,sltiu,sltu,sne,sqrt.d,sqrt.s,sra,srav,srl,%
    srlv,sub,sub.d,sub.s,subi,subiu,subu,sw,swc1,swl,swr,syscall,teq,%
    teqi,tge,tgei,tgeiu,tgeu,tlt,tlti,tltiu,tltu,tne,tnei,trunc.w.d,%
    trunc.w.s,ulh,ulhu,ulw,ush,usw,xor,xori},
  % assembler directives
  morekeywords={[2].align,.ascii,.asciiz,.byte,.data,.double,.extern,%
    .float,.globl,.half,.kdata,.ktext,.set,.space,.text,.word},
  % register names
  morekeywords={[3]\$0,\$1,\$2,\$3,\$4,\$5,\$6,\$7,\$8,\$9,\$10,\$11,%
    \$12,\$13,\$14,\$15,\$16,\$17,\$18,\$19,\$20,\$21,\$22,\$23,\$24,%
    \$25,\$26,\$27,\$28,\$29,\$30,\$31,%
    \$zero,\$at,\$v0,\$v1,\$a0,\$a1,\$a2,\$a3,\$t0,\$t1,\$t2,\$t3,\$t4,
    \$t5,\$t6,\$t7,\$s0,\$s1,\$s2,\$s3,\$s4,\$s5,\$s6,\$s7,\$t8,\$t9,%
    \$k0,\$k1,\$gp,\$sp,\$fp,\$ra},
    literate=
  {á}{{\'a}}1 {é}{{\'e}}1 {í}{{\'i}}1 {ó}{{\'o}}1 {ú}{{\'u}}1
  {Á}{{\'A}}1 {É}{{\'E}}1 {Í}{{\'I}}1 {Ó}{{\'O}}1 {Ú}{{\'U}}1
  {à}{{\`a}}1 {è}{{\`e}}1 {ì}{{\`i}}1 {ò}{{\`o}}1 {ù}{{\`u}}1
  {À}{{\`A}}1 {È}{{\'E}}1 {Ì}{{\`I}}1 {Ò}{{\`O}}1 {Ù}{{\`U}}1
  {ä}{{\"a}}1 {ë}{{\"e}}1 {ï}{{\"i}}1 {ö}{{\"o}}1 {ü}{{\"u}}1
  {Ä}{{\"A}}1 {Ë}{{\"E}}1 {Ï}{{\"I}}1 {Ö}{{\"O}}1 {Ü}{{\"U}}1
  {â}{{\^a}}1 {ê}{{\^e}}1 {î}{{\^i}}1 {ô}{{\^o}}1 {û}{{\^u}}1
  {Â}{{\^A}}1 {Ê}{{\^E}}1 {Î}{{\^I}}1 {Ô}{{\^O}}1 {Û}{{\^U}}1
  {œ}{{\oe}}1 {Œ}{{\OE}}1 {æ}{{\ae}}1 {Æ}{{\AE}}1 {ß}{{\ss}}1
  {ű}{{\H{u}}}1 {Ű}{{\H{U}}}1 {ő}{{\H{o}}}1 {Ő}{{\H{O}}}1
  {ç}{{\c c}}1 {Ç}{{\c C}}1 {ø}{{\o}}1 {å}{{\r a}}1 {Å}{{\r A}}1
  {€}{{\euro}}1 {£}{{\pounds}}1 {«}{{\guillemotleft}}1
  {»}{{\guillemotright}}1 {ñ}{{\~n}}1 {Ñ}{{\~N}}1 {¿}{{?`}}1
}[strings,comments,keywords]

\definecolor{CommentGreen}{rgb}{0,.6,0}
 \lstset{
   language=[mips]Assembler,
   backgroundcolor=\color{backcolour},   
   escapechar=\# @, % include LaTeX code between `@' characters
   keepspaces,   % needed to preserve spacing with lstinline
   basicstyle=\footnotesize\ttfamily\bfseries,
   numberstyle=\footnotesize\ttfamily\bfseries,
   commentstyle=\color{CommentGreen},
   stringstyle=\color{cyan},
   showstringspaces=false,
   captionpos=b,
   numbers=left,
   keywordstyle=[1]\color{blue},    % instructions
   keywordstyle=[2]\color{magenta}, % directives
   keywordstyle=[3]\color{red},     % registers
 }    

\lstset{language=[mips]{Assembler}}


%utilisation de la couleur définie avant
%toutes les sections auront cette couleur
\sectionfont{\color{havelockBlue}}
%\subsectionfont{\color{havelockBlue}}
%début du document




\begin{document}

\renewcommand{\labelitemi}{$\bullet$}
\renewcommand{\labelitemii}{$\cdot$}
\renewcommand{\labelitemiii}{$\diamond$}
\renewcommand{\labelitemiv}{$\ast$}

%début d'un titre
\begin{titlepage}
            %centre les éléments
	\centering
	
	{\scshape\LARGE \color{Monokaimagenta} Laboratoire \\  \par}
	
	%espace vertical de 1 mms
	\vspace{1cm}
	
	{\Large\itshape Sven Rouvinez \& Johanna Melly\par}
	
	%http://www.personal.ceu.hu/tex/spacebox.htm
	\vfill
	Professeur\par
	%met le texte en gras 
	\textbf{Carlos Andrés Peña} \par% ajoute une ligne 
	\vspace{1cm}
	Assistant\par
	\textbf{Gaëtan Matthey}
	
	\vfill

            %affiche la date actuelle
	{\large \today\par}
	
%fin de la page de titre
\end{titlepage}

\section{Objectifs du laboratoire}
Comprendre le fonctionnement d'un processeur dit "pipeliné", à l'aide d'un processeur PRODIS "pipeliné" fourni. Modifier le circuit pour ajouter des mécanismes de détection des aléas et d'arrêt du pipeline.
\subsection{Étape 1}
\subsubsection{Objectif}
Exécuter le programme fourni, et relever un chronogramme.
\subsubsection{Programme fourni}
\lstinputlisting[caption=main.S original,label=et1a]{../labo5/mains/mainEt1.S}
\subsubsection{Chronogramme relevé}
\begin{figure}[H]
    \centering
    \includegraphics[width=.8\textwidth]{src/CHRONO_ET1_COL2.png}
    \captionof{figure}{Chonogramme}
    \label{fig:chrono_et1_pic}
\end{figure}
En jaune, le résulat de l'instruction ADD (ligne 11 du progamme), en vert, le résultat de l'instruction LDRH (ligne 13), et en rouge, le résultat de l'instruction AND (ligne 23).
\subsection{Étape 2}
\subsubsection{Objectif}
Exécuter le programme fourni, relever un chronogramme. Repérer les erreurs dûes aux aléas et leurs causes. Modifier le programme en ajoutant des instructions NOP afin de supprimer les aléas.
\subsubsection{Programme fourni}
\lstinputlisting[caption=main.S original ,label=et1a]{../labo5/mains/mainEt2.S}
\subsubsection{Chronogramme}
\begin{figure}[H]
    \centering
    \includegraphics[width=.8\textwidth]{src/CHRONO_ET2_V2_COL.png}
    \captionof{figure}{Chonogramme}
    \label{fig:chrono_et2_pic}
\end{figure}

\paragraph{Légende des couleurs:}\mbox{}\\
\begin{itemize}[before=\ttfamily\bfseries]
   \item \colorbox{yellow}{add r4,r0,\#2}
   \item \colorbox{Salmon}{add r3,r4,r1}
   \item \colorbox{Cyan}{ldrh r1,[r4,\#3*2]}
   \item \colorbox{green}{sub r5,r1,\#5}
   \item \colorbox{Orchid}{mov r0,\#0xFF}
   \item \colorbox{Dandelion}{mov r5,\#0xFF}
   \item \colorbox{Green}{orr r0,r5}
\end{itemize}


Les résultats incorrects sont entourés en rouge.
Ces erreurs sont dûes à des aléas causées par des dépendances. Ce sont 3 opérations qui causent ces aléas:
\begin{itemize}
    \item ADD (ligne 10), l'aléa est de type Read After Write. En effet, il y a lecture du registre r4 juste après une instruction d'écriture dans le registre r4.
    \item SUB (ligne 13), l'aléa est de type Read After Write. En effet, il y a lecture du registre r1 juste après une instruction d'écriture dans le registre r1.
    \item ORR (ligne 31), l'aléa est de type Read After Write. En effet, il y a lecture du registre r5 juste après une instruction d'écriture dans le registre 5.
\end{itemize}
\subsubsection{Programme corrigé}
\lstinputlisting[caption=main.S corrigé,label=et1a]{../labo5/mains/mainEt2Mod.S}
Des instructions NOP ont été rajoutées avant l'înstruction ADD, l'instruction SUB et l'instruction ORR.
Voici un shéma qui illustre la correction des aléas grâce à l'instruction NOP:
\begin{figure}[H]
    \centering
    \includegraphics[width=.8\textwidth]{src/ZELE.jpeg}
    \captionof{figure}{Instructions NOP}
    \label{fig:nope_zele_pic}
\end{figure}
Les 3 instructions NOP permettent à la partie DECODE (ID) de la deuxième instruction ADD de débuter après la partie WRITE BACK (WB) de la première instruction ADD.
\subsubsection{Chronogramme}
\begin{figure}[H]
    \centering
    \includegraphics[width=.8\textwidth]{src/CHRONO_ET2_CORR_COL.png}
    \captionof{figure}{Chonogramme}
    \label{fig:chrono_et2_corr_pic}
\end{figure}
Pour la légende des couleurs, voir la liste plus haut \\
Les résultats sont maintenant corrects.
\subsection{Étape 3}
\subsubsection{Objectif}
Modifier le circuit afin de détecter les aléas et stopper le pipeline si nécessaire.
\end{document}